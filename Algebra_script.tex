\documentclass[12pt]{report}
% ------------- Graphic ------------------------
\usepackage{svg}
\usepackage{graphicx}
\graphicspath{{images/}{../images/}}
% ----------------------------------------------

\usepackage[utf8]{inputenc}
\usepackage[T1]{fontenc}
\usepackage{textcomp}
\usepackage[dutch]{babel}
\usepackage{amsmath, amssymb}
\usepackage{subfiles}
\usepackage[framed]{ntheorem}
\usepackage{framed}
\usepackage{extarrows}
\usepackage{enumitem}

\theoremnumbering{arabic}
\theoremstyle{plain}
\RequirePackage{latexsym}
\theoremsymbol{\ensuremath{_\Box}}
\theorembodyfont{\itshape}
\theoremheaderfont{\normalfont\bfseries}
\theoremseparator{}
\newframedtheorem{Theorem}{Theorem}
\newframedtheorem{theorem}{Theorem}
\newtheorem{Satz}{Satz}
\newtheorem{satz}{Satz}
\newtheorem{Proposition}[section]{Proposition}
\newtheorem{proposition}[section]{Proposition}
\newframedtheorem{Lemma}{Lemma}
\newframedtheorem{lemma}{Lemma}
\newtheorem{Korollar}{Korollar}
\newtheorem{korollar}{Korollar}
\newtheorem{Corollary}{Corollary}
\newtheorem{corollary}{Corollary}

\theorembodyfont{\upshape}
\newtheorem{Example}{Example}
\newtheorem{example}{Example}
\newtheorem{Beispiel}{Beispiel}
\newtheorem{beispiel}{Beispiel}
\newtheorem{Bemerkung}{Bemerkung}
\newtheorem{bemerkung}{Bemerkung}
\newtheorem{Anmerkung}{Anmerkung}
\newtheorem{anmerkung}{Anmerkung}
\newtheorem{Remark}{Remark}
\newtheorem{remark}{Remark}
\newframedtheorem{Definition}[section]{Definition}
\newframedtheorem{definition}[section]{Definition}

\theoremstyle{nonumberplain}
\theoremheaderfont{\scshape}
\theorembodyfont{\normalfont}
\theoremsymbol{\ensuremath{_\blacksquare}}
\RequirePackage{amssymb}
\newtheorem{Proof}{Proof}
\newtheorem{proof}{Proof}
\newtheorem{Beweis}{Beweis}
\newtheorem{beweis}{Beweis}
\qedsymbol{\ensuremath{_\blacksquare}}
\theoremclass{LaTeX}

\DeclareMathOperator{\Ima}{Im}
\DeclareMathOperator{\End}{End}

\title{
	{Algebra Script}\\
	{\large RWTH Aachen}\\
}

\author{Melkonian Dmytro}
\date{14 October 2018}

\begin {document}
	\maketitle

	\tableofcontents	
	
	\chapter{Gruppen, Ringe, Körper}
    \subfile{sections/section1-1.tex}
	
	\chapter{Matrizen und Lineare Gleichungen}
	\subfile{sections/section2-1.tex}
	
	\chapter{Vektorräume}
	\subfile{sections/section3-1.tex}

    \chapter{Bilinearformen, euklidische Räume und ihre komplexen Varianten}
    \subfile{sections/section10.tex}

    \chapter{Unitäre Abbildungen und Operatoren in Unitären Räumen}
    \subfile{sections/section11.tex}

    \chapter{Normalformen}
    \subfile{sections/section12.tex}

    \chapter{Ringe, Algebren, Moduln}
    \subfile{sections/section13.tex}

\end {document}
