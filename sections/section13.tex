\documentclass[../Algebra_script.tex]{subfiles}

\begin{document}
    \begin{definition}
        Ein \textbf{Ring} ist eine Menge $R$ mit zwei inneren Verknüpfungen $+, *$ so, dass $(R, +)$ eine abelshe Gruppe ist und $*$ eine assoziative
        Verknüpfung für $R$ mit einem neutralen Element (\textbf{Einselement}) ist. Es sollfür alle $a, b, c \in R$ gelten:
        \begin{align*}
            a*(b+c) &= a*b + a*c \\
            (b + c)*a &= b*a + c*a
        .\end{align*}
    \end{definition}

    \begin{definition}
        Es seien $R$ und $S$ zwei Ringe und $\phi : R \to S$ eine Abbildung. Dann heisst $\phi$ ein \textbf{Ringhomomorphismus} falls für alle $a, b, c \in R$
        gilt
        \begin{align*}
            \phi(a*b + c) = \phi(a)*\phi(b) + \phi(c) \; \text{und} \; \phi(1_{R}) = 1_{S}
        .\end{align*}
    \end{definition}

    \begin{definition}[Ideal]
        Es sei $R$ ein Ring und $I \subseteq R$ eine Untergruppe (bzgl. $+$ ). Dann heisst $I$ 
        \begin{itemize}
            \item ein \textbf{Linksideal} von $R$, falls für alle $r \in R$ und $a\in I: ra \in I$.
            \item ein \textbf{Rechtsideal} von R, falls für alle $r \in R$ und $a \in I: ar \in I$.
            \item ein \textbf{(beidseitiges) Ideal} von $R$, falls für alle $r \in R$ und $a \in I: ra \in I \wedge ar \in I$.
        \end{itemize}
    \end{definition}

    \begin{proposition}
        Es sei $\varphi: R \to S$ ein Ringhomomorphismus, dann ist $\ker \varphi$ ein Ideal in $R$. Umgekehrt sei $I \subseteq R$ ein Ideal, dann ist die
        kanonische Abbildung $\pi: R \to R\setminus I, r\mapsto \overline{r}$ ein Ringhomomorphismus.
    \end{proposition}

    \begin{definition}[Algebra]
        Es sei $K$ ein Körper. Ein $K$-Vektorraum $A$ heisst \textbf{Algebra} über $K$, falls es eine Abbildung gibt,
        \begin{align*}
            A \times A \to A, (a, b) \mapsto a*b
        .\end{align*}
        so, dass $(A, +, *)$ ein Ring mit Eins ist und für alle  $a,b \in A, \lambda \in K$gilt
        \begin{align*}
            \lambda(a*b) = (\lambda a)*b = a*(\lambda b)
        .\end{align*}
    \end{definition}

    \begin{definition}[Algebrahomomorphismus]
        Es seien $A_1,A_2$ jeweils $K$-Algebren. Es sei $\phi : A_1 \to A_2$ ein Vektorraumhomomorphismus. Dann heisst $\phi$ \textbf{Algebrenhomomorphismus}
        wenn $\phi$ auch ein Ringhomomorphismus ist.
    \end{definition}

    \begin{definition}[Modul]
        Es sei $R$ ein Ring mit Eins, $M$ eine abelshe Gruppe. Dann ist $M$ ein  $R$-\textbf{Linksmodul}, falls es ein Abbildung gibt
        \begin{align*}
            R \times M \to M, (r, m) \mapsto r.m
        .\end{align*}
        so, dass für alle $r, s \in R$ und für alle $m ,n \in M$ gilt
        \begin{align*}
            &(r*s).m = r.(s.m) \; \text{und} \; 1.m = m\\
            &(r + s).(m + n) = r.m + s.m + r.n + s.n
        .\end{align*}
        Entsprechend isr $R$ ein $R$-\textbf{Reschtsmodul}, falls es eine Abbildung gibt
        \begin{align*}
            M \times R, (m, r) \mapsto m.r
        .\end{align*}
        so, dass für alle $r,s \in R$ und für alle $n, m\ in M$ gilt
        \begin{align*}
            &m.(r*s) = (m.r).s \; \text{und} \; m = m.1\\
            &(m + n).(r + s) = m.r + m.s + n.r + n.s
        .\end{align*}
    \end{definition}

    \begin{definition}[Untermodul]
        Es sei $M$ ein $R$-Modul, eine Untergruppe $U \subseteq M$ heisst Untermodul von $M$, falls $\forall r \in R, u \in U: r.u \in U$.
    \end{definition}

    \begin{definition}[Modul-Homomorphismus]
        Es seien $N, M$ zwei $R$-Moduln und $\varphi: M \to N$ ein Gruppenhomomorphismus. Dann heisst $\varphi$ ein $R$-\textbf{Modul-Homomorphismus} genau
        dann, wenn
        \begin{align*}
            \forall m \in M, r \in R: \varphi(r.m) = r \varphi(m)
        .\end{align*}
        Die Menge der $R$-Modul-Homomorphismen beziechnen wir mit $\hom_{R}(M, N)$.
        Ein invertierbarer \textbf{Modul-Homomorphismus} heisst Isomorphismus, die Moduln $M$ und $N$ heisst dann \textbf{isomorph}
    \end{definition}

    \begin{definition}
        Es seien $M$ und $N$ zwei $R$-Moduln, dann wird $M \times N$ wieder zum einem $R$-Modul durch
        \begin{align*}
            r.(m, n) = (r.m, r.n)
        .\end{align*}
    \end{definition}

    \begin{definition}[direkte Summe]
        Es sei $M$ ein $R$-Modul, $U_1, U_2 \subseteq M$ $R$-Untermoduln, dann sagen wir $M$ ist \textbf{direkte Summe} von  $U_1$ und $U_2$, $M = U_1 \oplus
        U_2$, falls $U_1 \cap U_2 = 0$ und $U_1 + U_2 = M$.
    \end{definition}

    \begin{definition}
        Es sei $M \neq 0$ ein $R$-Modul. M heisst \textbf{unzerlegbar} falls für alle Untermoduln $U_1, U_2 \subseteq M$ gilt 
        \begin{align*}
            U_1 \oplus U_2 = M \implies U_1 = 0 \vee U_2 = 0
        .\end{align*}
        Anderfalls heisst $M$ zerlgebar.
    \end{definition}

    \begin{proposition}
        Es sei $U \subseteq M$ ein $R$-Untermodul, dann ist $M\setminus \ker \varphi$ isomoph zu $\Im \varphi$.

        Wir erhalten also eine kurze exacte Sequenz
        \begin{align*}
            0 \to \ker \varphi \to M \to \Im \varphi \to 0
        .\end{align*}
    \end{proposition}

    \begin{definition}[Endlich-erzeugte Moduln]
        Es sei $R$ ein Ring und $M$ ein $R$-Modul. Ein \textbf{Erzeugendensystem} von $M$ ist eine Teilmenge  $S = \{s_{i} | i \in I\} \subseteq M$ so, dass für
        jedes $m \in M$ existieren $\{r_{i} | i \in I\}$, wobei nur endlich vile $r_{i} \neq 0$ sind, mit
        \begin{align*}
            m = \sum_{i \in I} r_{i}s_{i}
        .\end{align*}
        $M$ heisst \textbf{endlich-erzuegt}, falls es ein endliches Erzeugendensystem gibt und \textbf{zyklisch}, falls es ein Erzuegendensystem $S$ gibt, mit
        $|S| = 1$.
    \end{definition}

    \begin{example}
        Sei $M$ eine $R$-Modul, $m_1, \ldots, m_{l} \in M$, dann ist $(m_1, \ldots, m_{l}) = \sum Rm_{i} \subseteq M$ 

        $R = \mathbb{Z}, m_1 = 3, m_2 = 2 \implies (2, 3) = 2\mathbb{Z} + 3\mathbb{Z} = \mathbb{Z} = (1)$ 

        $R = \mathbb{C}[x, y] (xy - 2, x + y) = \mathbb{C}[x,y](xy - 2) + \mathbb{C}[x,y](x + y)$

        $R$ als $R$-Linksmodul $\implies$ Untermodul $\simeq$ Linksideal
    \end{example}

    \begin{proposition}
        Es sei $R$ ein Ring, dann ist die Abbildug
        \begin{align*}
            \{(\text{Links-})\text{Ideale in}\; R\} \to \{\text{zyklische}\;(\text{Links-})\;\text{Moduln}\;\}\setminus\text{Isomorphie}
        \end{align*}
        eine Bijektion.
    \end{proposition}
    
    \begin{definition}[Freier Modul]
        Es sei $R$ ein Ring und $M$ ein $R$-Modul. Eine \textbf{Basis} von  $M$ ist ein linear unabhängiges Erzeugendensystem von $M$.
        Wenn $M$ eine Basis besitzt, so nennen wir $M$ einen \textbf{freien Modul} über $R$. Für jede Indexmenge $I$, schreiben wir $R^I$ für den freien
        $R$-Modul mit einer Basis indiziert durch $I$.
    \end{definition}

    \begin{proposition}
        Es sei $M$ ein $R$-Modul und $\{m_{i}|i\in I\}$ eine Teilmenge von $M$. Dann existiert genau ein Modulhomomorphismus
        \begin{align*}
            \pi: R^I \to M, e_{i} \mapsto m_{i} \; \text{für alle}\; i \in I
        .\end{align*}
        Das gilt insbesondere wenn $\{m_{i}| i \in I\}$ ein Erzeugendensystem von $M$ ist.
    \end{proposition}

    \begin{definition}[Kring]
        Einen kommutativen Ring mit $1$ nennen wir einen \textbf{Kring}.
    \end{definition}

    \begin{definition}
        Es sei $R \neq 0$ ein Ring. Dann heisst $R$ \textbf{nullteilerfrei}, wenn für alle $a, b \in R$ gilt:
        \begin{align*}
            a*b = 0 \implies a = 0 \vee b = 0
        .\end{align*}
        Ist $R$ darüber hinaus ein Kring, se nennen wir $R$ einen \textbf{Integritätsbereich}
    \end{definition}

    \begin{definition}
        Ein $R$ ein Integritätsberiech. $R$ heisst ein \textbf{euklidischer Ring}, falls eine Funktion $\delta: R \setminus \{0\} \to \mathbb{N}$ existiert mit:
        \begin{align*}
            \forall a, b \in R, a \neq 0 \exists q, r \in R: b = qa + r, r \neq 0 \implies \delta(r) < \delta(a)
        .\end{align*}
    \end{definition}

    \begin{definition}
        Es sei $R$ ein Ring, die Menge der \textbf{Einheiten} $R^*$ in $R$ ist die Menge der multiplikativ invertierbaren Elemente.
        $r \in R$ heisst \textbf{irreduzible} wenn $r$ keine Einheit hat und $r = fg$ impliziert, dass $f \in R^*$ oder $g \in R^*$.
    \end{definition}

    \begin{definition}
        Es sei $R$ ein Ring und $\{f_{i}| i \in I\} \subseteq R$. Dann bezeichnet $(\{f_{i} | i \in I\}) = \sum_{i \in I}^{} Rf_{i}R$ das von $\{f_{i} | i \in
        I\}$ \textbf{erzeugte Ideal} in R.
        $R$ ist ein Hauptidealring, wenn jedes Ideal in $R$ von einem Element erzeugt wird.
    \end{definition}


\end{document}
