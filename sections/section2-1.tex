\documentclass[../Algebra_script.tex]{subfiles}
\begin{document}

\begin{definition}{(Lineares Gleichungssystem)}
	Es sei $K$ ein Körper, $m, n \in \mathbb{N}, a_{ij}, b_i \in K$. Dann nennt man
	\[\begin{matrix}
		a_{1,1}x_1 &+ &a_{1,2}x_2 &+ &\ldots &+ &a_{1,n}x_n &= &b_1 \\
		a_{2,1}x_1 &+ &a_{2,2}x_2 &+ &\ldots &+ &a_{2,n}x_n &= &b_2\\
		 \vdots    &  &\vdots     &  &       & &\vdots     & &\vdots\\
		a_{m,1}x_1 &+ &a_{m,2}x_2 &+ &\ldots &+ &a_{m,n}x_n &= &b_m 
	\end{matrix}\]	
	ein \textbf{lineares Gleichungssystem (LGS)}, wobei die Menge aller $(x_1, \ldots, x_n) \in K^n$ gesucht ist, die alle Gleichungen erfüllen.	
\end{definition}

\begin{definition}{(Matrix)}
	Etwas kompakter: Für $n, m \in \mathbb{N}$ und $a_{ij} \in K$, nennt man
	\[A_{m,n} = 
		\begin{pmatrix}
			a_{1,1} & a_{1,2} & \cdots & a_{1,n} \\
			a_{2,1} & a_{2,2} & \cdots & a_{2,n} \\
			\vdots  & \vdots  & \ddots & \vdots  \\
			a_{m,1} & a_{m,2} & \cdots & a_{m,n} 
 		\end{pmatrix}
 	\]
 	eine $m \times n$-\textbf{Matrix}, die Zahlen $a_{ij}$ heissen \textbf{Einträge} oder \textbf{Elemente} der Matrix.
\end{definition}

\begin{definition}{(Operationen mit Matrizen)}
	Die Menge aller $m \times n$-Matrizen mit Einträgen in $K$ bezeichnen wir mit $M_{m,n}(K)$.
	\begin{enumerate}
		\item Es seien $A = (a_{ij})_{1 \leq i \leq m, 1 \leq j \leq n}, B = (b_{ij})_{1 \leq i \leq m, 1 \leq j \leq n} \in M_{m, n}(K)$. Dann definieren wir $A + B \in M_{m, n}(K)$ durch
		\[(A + B) := C = (c_{ij})_{1 \leq i \leq m, 1 \leq j \leq n} \text{ wobei } c_{ij} := a_{ij} + b_{ij} .\]
		\item Es seien $A \in M_{m, n}(K)$ und $B \in M_{n, \ell}(K)$. Dann definieren wir $A \cdot B \in M_{m, \ell}(K)$ durch
		\[(A \cdot B) := C = (c_{ij})_{1 \leq i \leq m, 1 \leq j \leq \ell} \text{ wobei } c_{ij} := \sum_{k=1}^{n} a_{ik}b_{kj} .\]  
	\end{enumerate}
\end{definition}

\begin{definition}
	Das lineare Gleichungssystem $Ax = b$ heisst \textbf{homogen}, falls $b = 0$, ansonsten heisst es \textbf{inhomogen}
\end{definition}

\begin{remark}
	Jedes homogene LGS besitzt immer die \textbf{triviale Lösung} $x = 0$. Wir suchen also vor allem Lösungen $x \neq 0$.
	
	Für ein LGS $Ax = b$ betrachten wir die \textbf{erweiterte Koeffizientenmatrix} $(A | b)$:
	\[(A | b) = 
		\left(\begin{array}{cccc|c}
			a_{1,1} & a_{1,2} & \cdots & a_{1,n} &b_1\\
			a_{2,1} & a_{2,2} & \cdots & a_{2,n} &b_2\\
			\vdots  & \vdots  & \ddots & \vdots  &\vdots\\
			a_{m,1} & a_{m,2} & \cdots & a_{m,n} &b_m
 		\end{array}\right)
 	\]
\end{remark}

\begin{definition}{(Gaussalgorithmus)}
	\textbf{Bestandteil des Gaussalgorithmus:}
	\begin{enumerate}
		\item Vorwärtselimination ($\to$ erreiche \textbf{Zeilenstufenform})
		\item Lösbarkeitsentscheidung
		\item Rückwärtssubstitution ($\to$ Unterscheidung \textbf{freie und abhängige Variable})
	\end{enumerate}
	\textbf{Zeilenstufenform}: Eine Matrix $A \in M_{m, n}(K)$ ist in Zeilenstufenform, wenn es eine Zahl $0 \leq r \leq m$ gibt, so dass
	\begin{itemize}
		\item in den ersten $r$-Zeilen jeweils nicht nur Nullen stehen und in den Zeilen $r+1$ bis $m$ nur Nullen stehen
		\item $j_1 < j_2 < \ldots < j_r$ wobei für $1 \leq i \leq r, j_i$ den minimale Index, so dass $a_{i,j_i} \neq 0$ ist.
	\end{itemize}
\end{definition}

\begin{proposition}
	Der Gaussalgorithm liefert nach endlich vielen Schritten entweder alle Lösungen des inhomogenen LGS oder endet mit einer negativen Entscheidung über Lösbarkeit des LGS.
\end{proposition}

Es sei $G$ eine abelesche Gruppe, dann ist $G^n$ auch eine abelesche Gruppe.

\begin{definition}
	Es sei $\End(G^n) = \{f : G^n \to G^n | f \text{ ist Gruppenhomomorphismus } \}$. Wir definiern
	\[+ : \End(G^n) \times \End(G^n) \to \End(G^n), \; (f_1, f_2) \mapsto \big(\underline{g} \mapsto f_1(\underline{g}) + f_2(\underline{g})\big)\]
	und
	\[\circ : \End(G^n) \times \End(G^n) \to \End(G^n), \; (f_1, f_2) \mapsto \big(\underline{g} \mapsto f_1(\underline{g})f_2(\underline{g})\big)\]
\end{definition}

\begin{proposition}
	$\End(G^n)$ ist ein Ring.
\end{proposition}

\begin{proposition}
	Die Menge $M_{n,n}(K)$ mit Addition und Multiplikation bildet einen Ring.
\end{proposition}
\end{document}